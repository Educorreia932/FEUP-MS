\begin{abstract}
Rush hour embarking and debarking at a transit platform can cause the intersection of large masses of people, making it harder for easy movement and arriving at the destination. A good design of a multi-modal transit station might help to attenuate this problem, bringing improvements in metrics such as mean average time to changing lines and, more subjectively, stress induced in passengers.
We develop and describe a simulation in Netlogo that is tested in the Trindade, Porto, multi-modal transit station. Multiple scenarios are considered to exemplify its usefulness in evaluating changes to the infrastructure.


\end{abstract}

\begin{IEEEkeywords}
modeling, system modeling, pedestrian traffic
\end{IEEEkeywords}