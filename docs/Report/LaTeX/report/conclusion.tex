\section{Conclusion} \label{sec:conclusion}
In this work, we used NetLogo to develop a simple pedestrian simulation of a transit station, with nothing more than an image as a guide for the modeling of said station. The implemented features would allow for the testing of most transit stations, given that common infrastructure was considered.
The results provided some promising results but were not sufficient to validate the model. Repetition of the experiments, a deeper analysis of the data, and other metrics could be employed to increase the value of the simulation.
In future work, passenger spawning could take a more realistic approach instead of passengers being all initialized at the start of the simulation. Additionally, path-finding algorithms that tackle the problems seen in \autoref{fig:straightened_path_around_obstacle} could be fruitful. Different types of agents, for example, agents with incomplete knowledge or different path preferences, were disregarded in this work. Studying other stations and how they compare in terms of metrics could provide insights into their relative performance. Despite some of the advantages of how \textit{turtles} and \textit{patches} are handled, NetLogo may not be the most suited tool for more extensive work in this area. Nonetheless, the methodology and algorithms developed may still provide guidance on how to implement a different system from scratch.