\section{Introduction} \label{sec:introduction}
% In urban landscapes, the convergence of large crowds during rush hours at transit platforms often results in challenges for efficient movement and timely arrival at destinations. The intricate dynamics of embarking and debarking passengers create bottlenecks that hinder the overall transit experience. This phenomenon not only affects quantitative metrics such as mean average time to change lines but also contributes to the subjective stress experienced by passengers. Addressing this issue requires a strategic approach to the design of multimodal transit stations. 

% This study aims to explore design alternatives for an idealized or specific multimodal transit station with a particular focus on minimizing alterations costs while maximizing improvements in passenger movement. We modelled Trindade Metro station, which is the central metro station in Porto, since is close to the city center, is the point of intersection of all the metro lines, and has access to several bus stations with different lines.

% By delving into the intricacies of station design, we seek to identify practical solutions that enhance the overall transit experience during peak hours, offering a harmonious balance between efficiency and minimalistic alterations.

Public transportation represents one of the most efficient ways to move large masses of people. Large multi-modal transit stations, where multiple modes of transportation coexist, have a lot of pedestrian traffic. Rush hour embarking and debarking between platforms, stops, or bays has large masses of people crossing each other and navigating the station space, often in directly opposite directions. It is in this overall context that a station with multiple floors serving multiple train platforms becomes an interesting problem, for it poses optimization problems not only in scheduling but also in the design of the station itself for faster traveling and cheaper maintenance and construction. \\
In the following work, we attempt to create a simulation of that scenario. We emphasize developing natural path-finding algorithms, particularly across multiple floors, in a computationally efficient way. Collisions between passengers are also taken into account. The developed model also attempts to generalize as much as possible, allowing for different scenarios to be tested easily and changing of parameters. \\

In \autoref{sec:related_work}, we mention some projects related to the pedestrian traffic problem, particularly in subway stations. In \autoref{sec:methods_materials}, we outline the main implementation details and thought process. In \autoref{sec:results_discussion}, we analyze multiple simulated scenarios by changing the environment and the running parameters. Finally, in \autoref{sec:conclusion}, we lay out the conclusions of the work and future work.
