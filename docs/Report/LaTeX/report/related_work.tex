\section{Related work} \label{sec:related_work}

With the increasing demand for public transportation, as city centers grow in population, subway stations have to adapt to accommodate the flow of all the passengers with minimal impediments.

As such, several projects exist regarding the simulation of pedestrian flow in stations as a means to analyze different phenomena and alternative scenarios before carrying out any changes to existing stations.

Some examples of successful stories include the Plaça de Catalunya Station \cite{catalunya}, which was remodeled according to a chosen design out of four after using the simulation software LEGION to analyze the most critical metrics. 

In the same fashion, a similar project was conducted for Moscow's metro \cite{moscow}, the busiest in Europe. A model was developed that reflected passenger flow from disembarking from one platform to another and calibrated using real-life data. Ultimately, a proposed solution to change the station topology was accepted.

Examples like these show how useful and insightful simulation and modeling are for optimizing traffic flow in railway stations. Their success boosts the validation of our project, and as such will be used as role models for our project.